\begin{center}
\textsc{\large Abstract}\\
To gain insights about the physical structure of protein domains, a Potts model is trained to represent pairwise statistical dependencies between amino-acids. For the first time, this Direct Coupling Analysis is performed within the framework of Imperatively Defined Factor Graphs. This enables the usage of the probabilistic programming framework \textsc{Factorie}. After laying out the problems of a maximum-likelihood based approach, the training is performed using Contrastive Divergence. The Contrastive Divergence gradients are used with the AdaGrad Stochastic Gradient Descent algorithm. To induce sparsity on the pairwise parameter estimate l1-regularization performed via Regularized Dual Averaging. The results are compared to a naïve mutual information based analysis.\\[2cm]
\textsc{Zusammenfassung}\\
In dieser Arbeit wird ein Pottsmodell verwendet, um Vorhersagen über die dreidimensionale Struktur eines Proteins zu treffen. Hierfür wird das Model an mehrere verschiedene Proteinsequenzen (Multiple Sequence Alingments) aus verschiedenen Organismen angepasst, um die Norm der so gefundenen Parameter als Schätzer für die physische Distanz der einzelnen Aminosäuren im Protein zu nehmen. Dieses Verfahren ist unter dem Namen Direkte Kupplungs Analyse (Direct Coupling Analysis) bekannt. Entscheidend für die Eignung der Pottsmodellparameter zur Modellierung von Aminosäuren Interaktionen, ist das Einführen eines Strafterms auf die \ell_1-Norm der Parameter (\ell_1-Regularization) während des Anpassungsprozesses. 
Traditionelle Methoden der Modellanpassung versagen bei dem verwendeten Pottsmodell, weswegen der kontrastive Divergenz (Contrastiv Divergence) Algorithmus verwendet wird. 
\end{center}
\clearpage